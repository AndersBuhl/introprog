%!TEX encoding = UTF-8 Unicode
%!TEX root = ../lect-week09.tex

%%%


%\begin{Slide}{TODO: Begrepp att förklara}
%  Tänk igenom ordningen:
%  \begin{itemize}
%    \item java switch, scala match ... 
%  \end{itemize}
%\end{Slide}
%
%
%\begin{Slide}{Javas switch-sats}
%A switch in Java works with the byte, short, char, and int primitive data types. It also works with enumerated types (discussed in Enum Types), the String class, and a few special classes that wrap certain primitive types: Character, Byte, Short, and Integer
%\end{Slide}


\ifkompendium\else

\Subsection{Nyhet: Scala 2.12.0}
\begin{Slide}{Nyhet: Scala 2.12.0 släpptes 3:e Nov 2016}
\begin{itemize}
\item Nytt i Scala 2.12:
\begin{itemize}\SlideFontSmall
\item \Emph{Optimeringar} ''under huven'' som \Alert{kräver Java 8}
\item \Emph{Snabbare, klarar sig med mindre minne, kortare bytekod, ...} 
\item Väsentligt förbättrad \Emph{Scaladoc}: \url{http://www.scala-lang.org/api}
\item Du hittar gamla Scaladoc för 2.11.8 här: \\ \url{http://www.scala-lang.org/api/2.11.8/}

\end{itemize}
\item I denna kursomgång och på LTH:s datorer kommer vi att stanna kvar vid \Emph{2.11.8} (nästa kursomgång kör vi 2.12) 
\item Observera att 2.12 \Alert{inte är bytekodskompatibel} med 2.11 så du måste kompilera om all gammal kod om den ska funka med nykompilerad kod om du installerar 2.12.
\item Med \code{sbt} (se appendix G) är det enkelt att ha många olika versioner av Scala-kompilatorn igång på samma maskin.
\end{itemize}
{\SlideFontTiny För den intresserade, läs mer här: \url{http://www.scala-lang.org/news/2.12.0}}
\end{Slide}


\Subsection{Veckans lab: \texttt{chords-team}}
\begin{Slide}{Veckans lab: \texttt{chords-team}}\SlideFontSmall
Övergripande syfte:
\begin{itemize}
\item Träna på case-klasser, matchning, undantag
\item Jobba med ett större program med flera klasser i olika filer
\item Jobba flera personer på samma program
\end{itemize}
Innehåll:
\begin{itemize}
\item Skapa och spara ackord på gitarr (6 strängar) och ukulele (4 strängar) 
\item Spela upp ackord med Javas inbyggda midispelare inkapslad i \code{SimpleNotePlayer}
\item Rita ackord med \code{SimpleWindow} 
\end{itemize}
Hur mycket ni gör beror på hur många ni är i gruppen och hur stora ambitioner ni har. Diskutera detta med handledare på resurstid.
\end{Slide}

\begin{Slide}{Toner, oktaver och ackord}\SlideFontSmall
\begin{itemize}
\item Det finns 12 toner som har speciella namn: \\ C, C\#, D, D\#, etc. (uttalas: c, ciss, d, diss, etc.)
\item Jämför vita och svarta tangenter på ett piano: \\ avståndet mellan varje tangent är ett s.k. \emph{halvt tonsteg}. 
\item Toner återkommer i oktaver, modulo 12.
\item Tonen som representeras av strängen \code{"D2"} är tonen D i andra oktaven.
\item Tonen \code{"D2"} motsvarar heltalet \code{26} på labben.
\item Ett ackord består av flera toner.
\end{itemize}
\begin{REPL}[basicstyle=\color{white}\ttfamily\SlideFontSize{6}{7}\selectfont]
scala> val notes = Vector("C", "C#", "D", "D#", "E", "F", "F#", "G", "G#", "A", "A#", "B")

scala> notes.size
res0: Int = 12

scala> notes(26 % 12)
res1: String = D

\end{REPL}
\end{Slide}

\begin{Slide}{Toner på ett stränginstrument}
\begin{minipage}{0.5\textwidth}
\begin{itemize}\SlideFontSmall
\item Gitarr och ukulele har 6 resp. 4 strängar och en greppbräda med s.k. band.

\item Om man trycker ned ett finger på första bandet höjs tonen ett halvt tonsteg. 

\item Exempel: om en sträng är stämd i D3 blir tonen om man trycker ned fingret på fjärde bandet F\#3.

\item \href{http://www.gitarr.org}{www.gitarr.org}

\item \href{http://www.stefansukulele.com}{www.stefansukulele.com}

\end{itemize}
\end{minipage}
\begin{minipage}{0.45\textwidth}
\includegraphics[width=1.0\textwidth]{../img/chords/ChordDraw}
\end{minipage}

\end{Slide}

\begin{Slide}{Modell av gitarr och ukulele}
\begin{Code}
object model {
  type Tuning = Vector[String]
  type Grip = Vector[Int]

  trait Chord {
    def name: String
    def tuning: Tuning
    def grip: Grip
  }
  
  case class Guitar(name: String, grip: Grip) extends Chord {
    val tuning = Vector("E2", "A2", "D3", "G3", "B3", "E4")
  }
  
  case class Ukulele(name: String, grip: Grip) extends Chord {
    val tuning = Vector("G4", "C4", "E4", "A4")
  }

}
\end{Code}
\end{Slide}

\begin{Slide}{En gemensam bastyp för olika ackord}\SlideFontSmall
\vspace{-0.5em}\begin{center}
\newcommand{\TextBox}[1]{\raisebox{0pt}[1em][0.5em]{#1}}
\tikzstyle{umlclass}=[rectangle, draw=black,  thick, anchor=north, text width=3cm, rectangle split, rectangle split parts = 3]
\begin{tikzpicture}[inner sep=0.5em]
\node [umlclass, rectangle split parts = 2, xshift=0cm, text width=5cm] (BaseType)  {
            \textit{\textbf{\centerline{\TextBox{\code{Chord}}}}}
            \nodepart[]{second}
            \TextBox{\code{val name: String}}\vspace{-0.25em}\newline
            \TextBox{\code{val tuning: Vector[String]}}\vspace{-0.25em}\newline
            \TextBox{\code{val grip: Vector[Int]}}\vspace{-0.25em}\newline

        };
        
\node [umlclass, rectangle split parts = 1]  at (2.5cm,-3.7cm) (SubType1) {
            \textbf{\centerline{\TextBox{\code{Guitar}}}}
            %\nodepart[]{second} \TextBox{~}
        };  
                
\node [umlclass, rectangle split parts = 1] at (-2.5cm,-3.7cm) (SubType2)  {
            \textbf{\centerline{\TextBox{\code{Ukulele}}}}
            %\nodepart[]{second} \TextBox{talk(): void}
        };        
\draw[umlarrow] (SubType1.north) -- ++(0,0.5) -| (BaseType.south);    
\draw[umlarrow] (SubType2.north) -- ++(0,0.5) -| (BaseType.south);            
\end{tikzpicture}
\end{center}
\pause\vspace{-0.7em}
\begin{REPL}
scala> import model._

scala> val uc = Ukulele("C", Vector(0, 0, 0, 3))
uc: model.Ukulele = Ukulele(C,Vector(0, 0, 0, 3))

scala> val ge = Guitar("E", Vector(0, 2, 2, 1, 0, 0))
ge: model.Guitar = Guitar(E,Vector(0, 2, 2, 1, 0, 0))
\end{REPL}
\end{Slide}


\begin{Slide}{Grupparbete}\SlideFontSmall
\begin{itemize}
\item Förslag på arbetssätt:
\begin{itemize}\SlideFontSmall
\item Träffas nu på rasten och boka nästa gruppmöte
\item Förberedelser inför första gruppmötet: individuella studier av labbinstruktioner och koden som är given i workspace
\item Träffas gärna i ett studierum med whiteboard
\item På mötet: gå igenom uppgift och given kod så att alla fattar vad det går ut på; bestäm omfattning och ansvarsuppdelning
\item När ni träffas, skissa upp din kod på whiteboard och få feedback
\item På varje gruppmöte, bestäm tid för nästa möte och vad var och en ska försöka hinna tills dess
\end{itemize}


\item Ni får \Emph{lov att ändra på omfattningen} efter antalet gruppmedlemmar, ambition och förmåga: diskutera detta med handledare på resurstid

\item \Alert{Minimikrav}: att med textkommando kunna skapa/spara/ladda gitarr- och ukulele-ackord och att ni tränar på matchning

\end{itemize}
\end{Slide}


\Subsection{Matchning}
\begin{Slide}{Vad är matchning?}
Matchning gör man då man vill jämföra ett värde mot andra värden och hitta överensstämmelse \Eng{match}.

\pause

\vspace{1em}Detta kan man t.ex. göra med nästlade if-else-satser:

\begin{Code}
val g = scala.io.StdIn.readLine("grönsak:")

if (g == "gurka") println("gott!")
else if (g == "tomat") println("gott!")
else if (g == "broccoli") println("ganska gott...")
else println("inte gott :(")
\end{Code}
\end{Slide}
\fi

\begin{Slide}{Javas switch-sats}\SlideFontSmall
De flesta C-liknande språk (men inte Scala) har en \jcode{switch}-sats som man kan använda istället för (vissa) nästlade if-else-satser: 
\javainputlisting[basicstyle=\ttfamily\SlideFontSize{6}{7}\selectfont]{../compendium/examples/match/SwitchNoBreak.java}
Funkar bara för primitiva typer och några till (t.ex. String).
\end{Slide}

\ifkompendium\else
\begin{Slide}{Javas switch-sats utan break}\SlideFontSmall
Saknad \jcode{break}-sats ''faller igenom'' till efterföljande gren: 

\javainputlisting[basicstyle=\ttfamily\SlideFontSize{6}{7}\selectfont]{../compendium/examples/match/SwitchNoBreak.java}
En glömd \jcode{break} kan ge svårhittad bugg... 
\end{Slide}

\begin{Slide}{Javas switch-sats med glömd break}\SlideFontSmall

\vspace{-0.5em}\javainputlisting[basicstyle=\ttfamily\SlideFontSize{5.5}{6.8}\selectfont]{../compendium/examples/match/SwitchForgotBreak.java}

\vspace{-0.7em}\pause
\begin{REPL}
$ java SwitchForgotBreak 
Skriv grönsak:
gurka
gott!
gott!
\end{REPL}

\end{Slide}


\begin{Slide}{Scalas \texttt{match}-uttryck}
Scala har ingen \code{switch}-sats men erbjuder i stället ett \code{match}-\Emph{uttryck} som är mångsidig och kraftfull och ger ett värde.

\begin{Code}
val g = scala.io.StdIn.readLine("grönsak:")
val msg = g match {
  case "gurka" => "gott!"
  case "tomat" => "jättegott!"
  case "broccoli" => "ganska gott..."
  case _ => "mindre gott..."
}
\end{Code}
Och den ''faller inte igenom'' som Javas \code{switch}-sats! \\
Default-grenen skrivs så här: \code{case _ => }
\end{Slide}

\begin{Slide}{Matchning med gard}
\end{Slide}

\begin{Slide}{Matchning efter typ}
\end{Slide}

\begin{Slide}{Stora/små begynnelsebokstäver vid matchning}
\end{Slide}


\fi





