%!TEX encoding = UTF-8 Unicode
%!TEX root = ../lect-week08.tex

%%%

\ifkompendium\else

\Subsection{Resultat på kontrollskrivnig}
\begin{Slide}{Resultat på kontrollskrivning 2016}
\pgfplotstableread[row sep=\\,col sep=&]{
    poäng & kamraträttat & korrigerat\\
    0     & 14  & 11\\
    1     & 28  & 28 \\
    2     & 16  & 14\\
    3     & 20  & 21 \\
    4     & 22  & 19 \\
    5     & 7   & 14 \\
    }\mydata

\begin{minipage}{0.6\textwidth}
\hspace*{-0.65cm}\begin{tikzpicture}[scale=0.9, every node/.style={scale=0.9}]
    \begin{axis}[
            ybar,
            symbolic x coords={0,1,2,3,4,5},
            xtick=data,
            nodes near coords,
            nodes near coords align={vertical},
            legend style={at={(0.5,1)},anchor=south,legend columns=-1,draw=none},
            ymin=0,ymax=35,
            ylabel={Antal},
            xlabel={Poäng},
        ]
        \addplot table[x=poäng,y=kamraträttat]{\mydata};
        \addplot table[x=poäng,y=korrigerat]{\mydata};
        \legend{kamraträttat, korrigerat}
    \end{axis}
\end{tikzpicture}
\end{minipage}%
\begin{minipage}{0.35\textwidth}
\begin{itemize}\SlideFontTiny
\item[] Totalt: 107 st (100\%)
\item[] 4 -- 5: 33st (31\%)
\item[] 3 -- 5: 54st (50\%)
\item[] 0 -- 2: 53st (50\%)
\item[] 0 -- 1: 39st (36\%)
\end{itemize}
\end{minipage}%


\end{Slide}


\Subsection{REBOOT CAMP}

\begin{Slide}{REBOOT CAMP}
\huge 

3-5: GRATTIS! Bli ännu starkare!

0-2: Fixa trösklar och luckor!

\vspace{0.5em} \Emph{STAY CALM} \\\vspace{0.5em} \Alert{GET ON TRACK}
\end{Slide}


\begin{Slide}{Omplanering: w08 = REBOOT CAMP}\SlideFontSmall
Det är \Alert{för många} som ligger \Alert{för långt efter}: \\
\Emph{Vi måste göra något!}
\begin{itemize}
\item Omplanering: w08 = REBOOT CAMP
\begin{itemize}\SlideFontTiny
\item \Alert{GE JÄRNET} för att stärka dig inför resten av kursen!
\item Noggrann genomgång av kontrollskrivning
\item Gör självdiagnostik och kämpa dig över trösklar och fyll igen luckor
\item Slipa dina inlärningsverktyg! 
\end{itemize}
\item Vi senarelägger alla kvarvarande labbar en vecka så att w08 frigörs;
 lab \code{chords-team} görs alltså i vecka w09 etc.

\item Sista labben \code{life} omdefinieras till att ingå bland projektalternativen i slutet av kursen (man får ändå öva på matriser på lab \code{maze})

\item Stoffet i veckorna w12 \& w13 slås ihop och minskas ned

\item Övn threads blir frivilligt extramaterial och ingår ej i examinationen.

\end{itemize}
\end{Slide}

\Subsection{Genomgång kontrollskrivnig}
\begin{Slide}{Genomgång av kontrollskrivning}
\end{Slide}


\Subsection{Slipa verktygen}
\begin{Slide}{Slipa verktygen}
Själdiagnostik
\begin{itemize}
\item Hur lär jag mig bäst?
\item Vad behöver jag extra tränig på?
\item Hur ska jag planera REBOOT CAMP?
\end{itemize}
\end{Slide}


\fi












