%!TEX encoding = UTF-8 Unicode
%!TEX root = ../lect-week08.tex

%%%

\ifkompendium\else

\Subsection{Resultat på kontrollskrivnig}
\begin{Slide}{Resultat på kontrollskrivning 2016}
\pgfplotstableread[row sep=\\,col sep=&]{
    poäng & kamraträttat & korrigerat\\
    0     & 14  & 11\\
    1     & 28  & 28 \\
    2     & 16  & 14\\
    3     & 20  & 21 \\
    4     & 22  & 19 \\
    5     & 7   & 14 \\
    }\mydata

\begin{minipage}{0.6\textwidth}
\hspace*{-0.65cm}\begin{tikzpicture}[scale=0.9, every node/.style={scale=0.9}]
    \begin{axis}[
            ybar,
            symbolic x coords={0,1,2,3,4,5},
            xtick=data,
            nodes near coords,
            nodes near coords align={vertical},
            legend style={at={(0.5,1)},anchor=south,legend columns=-1,draw=none},
            ymin=0,ymax=35,
            ylabel={Antal},
            xlabel={Poäng},
        ]
        \addplot table[x=poäng,y=kamraträttat]{\mydata};
        \addplot table[x=poäng,y=korrigerat]{\mydata};
        \legend{kamraträttat, korrigerat}
    \end{axis}
\end{tikzpicture}
\end{minipage}%
\begin{minipage}{0.35\textwidth}
\begin{itemize}\SlideFontTiny
\item[] Totalt: 107 st (100\%)
\item[] 4 -- 5: 33st (31\%)
\item[] 3 -- 5: 54st (50\%)
\item[] 0 -- 2: 53st (50\%)
\item[] 0 -- 1: 39st (36\%)
\end{itemize}
\end{minipage}%


\end{Slide}


\Subsection{REBOOT CAMP}

\begin{Slide}{REBOOT CAMP}
\huge 

3-5: GRATTIS! Bli ännu starkare!

0-2: Fixa trösklar och luckor!

\vspace{0.5em} \Emph{STAY CALM} \\\vspace{0.5em} \Alert{GET ON TRACK}
\end{Slide}


\begin{Slide}{Omplanering: w08 = REBOOT CAMP}\SlideFontSmall
Det är \Alert{för många} som ligger \Alert{för långt efter}: \\
\Emph{Vi måste göra något!}
\begin{itemize}
\item Omplanering: w08 = REBOOT CAMP
\begin{itemize}\SlideFontTiny
\item \Alert{GE JÄRNET} för att stärka dig inför resten av kursen!
\item Noggrann genomgång av kontrollskrivning
\item Gör självdiagnostik och kämpa dig över trösklar och fyll igen luckor
\item Slipa dina inlärningsverktyg! 
\end{itemize}
\item Vi senarelägger alla kvarvarande labbar en vecka så att w08 frigörs;
 lab \code{chords-team} görs alltså i vecka w09 etc.

\item Sista labben \code{life} omdefinieras till att ingå bland projektalternativen i slutet av kursen (man får ändå öva på matriser på lab \code{maze})

\item Stoffet i veckorna w12 \& w13 slås ihop och minskas ned

\item Övn threads blir frivilligt extramaterial och ingår ej i examinationen.

\end{itemize}
\end{Slide}


\Subsection{Slipa verktygen}

\begin{Slide}{Slipa verktygen}
För dig som har det \Alert{svårt}:
\begin{itemize}
\item Man kan inte lära sig ett språk bara genom att passivt läsa
\item Om du inte börjat än: nu måste du verkligen börja skriva, prata, uppfinna, konstruera, göra själv, vara aktiv, ...
\end{itemize}
För dig som har det \Emph{lätt}:
\begin{itemize}
\item Om du utmanar dig når du \Emph{mycket} längre
\item Analysera dina styrkor och svagheter
\item Utveckla din studieteknik och problemlösningsförmåga
\end{itemize}
\end{Slide}

\begin{Slide}{Vad avgör studieframgång?}
Studieteknik, Attityd till sina studier, (Talang)
\url{https://www.youtube.com/watch?v=gSbpRjxYq24}

\vspace{2em} Att repetera:
\url{https://www.youtube.com/watch?v=mmAmsaRH_VA}

\vspace{2em} Att planera:
\url{https://www.youtube.com/watch?v=g2BTFzYnNNY}
\end{Slide}

\begin{Slide}{Självdiagnostik och planering}
\begin{itemize}
\item Hur lär jag mig bäst?
\item Vad behöver jag extra träning på?
\begin{itemize}
\item Vad hade jag lätt resp. svårt för på kontrollskrivingen?
\item Vilka är mina \Alert{trösklar}? Extra svårt?
\item Vilka är mina \Alert{luckor}? Inte provat alls?
\item Vilka är mina \Emph{intressen}? Hur fördjupa mig?
\end{itemize}
\item Hur ska jag planera min REBOOT CAMP?
\begin{itemize}
\item Gör ett schema dag för dag. 
\item Vilken undervisning ska jag gå på?
\item Du som fick 0-2: gå på minst 2 resurstider. 
\item Hur mycket fritid kan jag frigöra till REBOOT CAMP?
\end{itemize}
\end{itemize}
\end{Slide}



\begin{Slide}{Strategier för problemlösning i programmering}
\begin{itemize}
\item Börja med ett litet men fungerande program; ta sedan många små steg och testa hela tiden att det fungerar
\item Om problemet är för \Alert{svårt}:\\ lös först ett \Emph{lättare}, relaterat problem 
\item Dela upp problemet i delar
\begin{itemize}
\item \code{val braNamn = delresultat}
\item \code{def delLösning = algoritm som löser delproblem}
\item \code{??? // inte klart än}
\end{itemize}
\item Problemlösning är inte linjärt: du måste kunna knåpa på ditt program i olika ''ändar''; skriva lite här och där; stoppa in; flytta runt; ändra
\end{itemize}
\end{Slide}

\begin{Slide}{Strategier för att komma över trösklar}\SlideFontSmall
\Alert{tröskel} == jag har svårt att begripa och komma vidare; kan ej själv konstruera

\begin{itemize}\SlideFontTiny
\item Du måste först \Emph{identifiera tröskeln} och tydligt formulera vad du inte förstår eller inte kan klara av att själv skapa.

\item Du måste hitta ett sätt att \Emph{konkretisera} begrepp och \Emph{visualisera} vad som händer \\
Använd analogier: kaffekvarnen för funktion, stämpla för instansiering, etc.

\item Använd flera exempel på samma sak: försök se \Emph{mönster} \\
Exempel: Tomat och Gurka är Grönsak; Student och Lärare är Person. \\ Lär dig pseudokodexempel på vanliga algoritmer i kompendiet utantill!

\item Gör \Emph{enklast möjliga} exempel som du exekverar: \\ Skapa en enkel klass med bara en heltalsmedlem och ''lek'' med den.

\item Bygg vidare på det du lär dig och \Emph{utvidga} stegvis med större exempel.\\ Exekvera allt större kod som du själv skriver!

\item \Emph{Avancera}: Kombinera med begrepp du redan känner. Exekvera!

\end{itemize}
Utgå från det du vet om hur just \Emph{du} lär dig bäst. Hur ska du vara \Alert{aktiv}? \\ Rita. Prata. Skriv sammanfattningar. Skapa egna program. ...
\end{Slide}

\Subsection{Genomgång kontrollskrivnig}
\begin{Slide}{Genomgång av kontrollskrivning}
\begin{itemize}
\item Förstå uppgiften

\item Strategi för lösning

\item Skapa lösning iterativt

\item Kontrollera lösning
\end{itemize}
\end{Slide}

\begin{Slide}{Uppdrag under rasten}
\begin{itemize}
\item Tala med med en eller två som är ungefär på din nivå med ledning av resultatet på kontrollskrivningen. (Eller skriv ner för dig själv om du helst vill vara ensam)

\item 5 minuter var: berätta för den andre om... 
\begin{itemize}
 \item dina trösklar: vad är extra svårt?
 \item dina luckor: vad har jag inte ens provat själv?
 \item dina fördjupningsintressen: vad vill jag veta mer om?
 \item övningar och laborationer som behöver kompletteras
\end{itemize}
\item \Alert{Fastna inte} i orsaker/ursäkter till situationen: \Emph{utgå från nuläget} och  indentifiera trösklar/luckor/fördjupning
\end{itemize}
\end{Slide}


\begin{Slide}{Tillbaka efter rasten:}
Påbörja detta arbete som du sedan fortsätter med i eftermiddag/kväll:  
\begin{itemize}
\item För varje begreppslista i w01-w07:
\begin{itemize}
\item Välj ut några begrepp som är viktiga för dig att träna mer på.
\item Välj ut några övningar som är kopplade till begreppen.
\item Gör en prioriteringsordning för begreppen/övningarna.
\item Planera ditt arbete för veckan:
\begin{itemize}
\item Övningar
\item Ev. labbar att komplettera
\end{itemize}
\end{itemize}
\end{itemize}
\Emph{Ta med dig priolistan} till \Alert{morgondagens föreläsning!}
\end{Slide}

\fi












