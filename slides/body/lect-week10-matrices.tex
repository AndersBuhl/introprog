%!TEX encoding = UTF-8 Unicode
%!TEX root = ../lect-week10.tex

%%%

\ifkompendium\else

\Subsection{Veckans labb: \texttt{maze}}
\begin{Slide}{Veckans labb: \texttt{maze}}\SlideFontSmall
Grunduppgift:
\begin{itemize}
\item Implementera en algoritm som hittar ut ur en labyrint.

\item En labyrint representeras av en \Emph{matris}, \\närmare bestämt en \Emph{vektor av vektorer} med  \Alert{booelska} värden: \\ \code{Vector[Vector[Boolean]]} 

\pause Där de två olika sanningsvärdena representerar följande:
\begin{itemize}\SlideFontSmall
\item \code{true} om det \Emph{finns en vägg} på en viss plats i matrisen
\item \code{false} om det \Alert{inte} finns en vägg på en viss plats i matrisen 

\end{itemize}
\pause
\item Använd enkel idé (som inte ger kortaste vägen): \\ Behåll vänster hand i kontakt med väggen och gå tills du når utgången.

\item Vad krävs av labyrinten för att detta ska fungera?  
\end{itemize}
\pause Extrauppgift:
\begin{itemize}
\item Generera slumpmässig labyrint 
\item Algoritmen (\emph{Prims algoritm}) är given i pseudokod
\end{itemize}

\end{Slide}

\begin{Slide}{Labyrint som booelsk matris}
\end{Slide}

\Subsection{Matriser}

\begin{Slide}{Vad är en matris?}
\begin{multicols}{2}
Hej
\columnbreak

\begin{itemize}
\pause
\item Hej

\end{itemize}
\end{multicols}
\end{Slide}


\fi











