%!TEX encoding = UTF-8 Unicode
%!TEX root = ../lect-week09.tex

%%%


\ifkompendium\else

\Subsection{Undantag}
\begin{Slide}{Vad är ett undantag \Eng{exception}?}
Undantag representerar ett fel eller ett onormalt tillstånd som upptäcks under exekvering och som  behöver hanteras på särskilt sätt vid sidan av det normala exekveringsflödet. 

\vspace{1em}\href{https://sv.wikipedia.org/wiki/Undantagshantering}{sv.wikipedia.org/wiki/Undantagshantering}


\vspace{1em} Exempel på undantag:

\pause

\begin{itemize}
\item Indexering utanför vektorns indexgränser.

\item Läsning bortom filens slut.

\item Försök att öppna en fil som inte finns.

\item Minnet är slut.

\item Division med noll.

\item \code{"hej".toInt} resulterar i \\\code{java.lang.NumberFormatException}

\end{itemize}

\end{Slide}


\begin{Slide}{Fånga undantag med \texttt{try}-\texttt{catch}-uttryck}
\end{Slide}

\Subsection{\texttt{scala.util.Try}}

\begin{Slide}{En gemensam bastyp för något som kan misslyckas}\SlideFontSmall
\vspace{-0.5em}\begin{center}
\newcommand{\TextBox}[1]{\raisebox{0pt}[1em][0.5em]{#1}}
\tikzstyle{umlclass}=[rectangle, draw=black,  thick, anchor=north, text width=3.0cm, rectangle split, rectangle split parts = 3]
\begin{tikzpicture}[inner sep=0.5em]
\node [umlclass, rectangle split parts = 2, xshift=0cm, text width=3.8cm] (BaseType)  {
            \textit{\textbf{\centerline{\TextBox{\code{Try[T]}}}}}
            \nodepart[]{second}
            \TextBox{\code{def get: T}}\newline
            \TextBox{\code{def isFailure: Boolean}}\newline
            \TextBox{\code{def isSuccess: Boolean}}
        };
        
\node [umlclass, rectangle split parts = 2, text width=2.2cm]  at (-2.5cm,-3.7cm) (SubType1) {
            \textbf{\centerline{\TextBox{\code{Success[T]}}}}
            \nodepart[]{second} \TextBox{\code{val value: T}}
        };  
                
\node [umlclass, rectangle split parts = 2, text width=4.2cm] at (2.5cm,-3.7cm) (SubType2)  {
            \textbf{\centerline{\TextBox{\code{Failure[T]}}}}
            \nodepart[]{second} \TextBox{\code{val exception: Throwable}}
        };        
\draw[umlarrow] (SubType1.north) -- ++(0,0.5) -| (BaseType.south);    
\draw[umlarrow] (SubType2.north) -- ++(0,0.5) -| (BaseType.south);            
\end{tikzpicture}
\end{center}
\end{Slide}

\fi





